% Options for packages loaded elsewhere
\PassOptionsToPackage{unicode}{hyperref}
\PassOptionsToPackage{hyphens}{url}
%
\documentclass[
]{article}
\author{}
\date{\vspace{-2.5em}}

\usepackage{amsmath,amssymb}
\usepackage{lmodern}
\usepackage{iftex}
\ifPDFTeX
  \usepackage[T1]{fontenc}
  \usepackage[utf8]{inputenc}
  \usepackage{textcomp} % provide euro and other symbols
\else % if luatex or xetex
  \usepackage{unicode-math}
  \defaultfontfeatures{Scale=MatchLowercase}
  \defaultfontfeatures[\rmfamily]{Ligatures=TeX,Scale=1}
\fi
% Use upquote if available, for straight quotes in verbatim environments
\IfFileExists{upquote.sty}{\usepackage{upquote}}{}
\IfFileExists{microtype.sty}{% use microtype if available
  \usepackage[]{microtype}
  \UseMicrotypeSet[protrusion]{basicmath} % disable protrusion for tt fonts
}{}
\makeatletter
\@ifundefined{KOMAClassName}{% if non-KOMA class
  \IfFileExists{parskip.sty}{%
    \usepackage{parskip}
  }{% else
    \setlength{\parindent}{0pt}
    \setlength{\parskip}{6pt plus 2pt minus 1pt}}
}{% if KOMA class
  \KOMAoptions{parskip=half}}
\makeatother
\usepackage{xcolor}
\IfFileExists{xurl.sty}{\usepackage{xurl}}{} % add URL line breaks if available
\IfFileExists{bookmark.sty}{\usepackage{bookmark}}{\usepackage{hyperref}}
\hypersetup{
  hidelinks,
  pdfcreator={LaTeX via pandoc}}
\urlstyle{same} % disable monospaced font for URLs
\usepackage[margin=1in]{geometry}
\usepackage{color}
\usepackage{fancyvrb}
\newcommand{\VerbBar}{|}
\newcommand{\VERB}{\Verb[commandchars=\\\{\}]}
\DefineVerbatimEnvironment{Highlighting}{Verbatim}{commandchars=\\\{\}}
% Add ',fontsize=\small' for more characters per line
\usepackage{framed}
\definecolor{shadecolor}{RGB}{248,248,248}
\newenvironment{Shaded}{\begin{snugshade}}{\end{snugshade}}
\newcommand{\AlertTok}[1]{\textcolor[rgb]{0.94,0.16,0.16}{#1}}
\newcommand{\AnnotationTok}[1]{\textcolor[rgb]{0.56,0.35,0.01}{\textbf{\textit{#1}}}}
\newcommand{\AttributeTok}[1]{\textcolor[rgb]{0.77,0.63,0.00}{#1}}
\newcommand{\BaseNTok}[1]{\textcolor[rgb]{0.00,0.00,0.81}{#1}}
\newcommand{\BuiltInTok}[1]{#1}
\newcommand{\CharTok}[1]{\textcolor[rgb]{0.31,0.60,0.02}{#1}}
\newcommand{\CommentTok}[1]{\textcolor[rgb]{0.56,0.35,0.01}{\textit{#1}}}
\newcommand{\CommentVarTok}[1]{\textcolor[rgb]{0.56,0.35,0.01}{\textbf{\textit{#1}}}}
\newcommand{\ConstantTok}[1]{\textcolor[rgb]{0.00,0.00,0.00}{#1}}
\newcommand{\ControlFlowTok}[1]{\textcolor[rgb]{0.13,0.29,0.53}{\textbf{#1}}}
\newcommand{\DataTypeTok}[1]{\textcolor[rgb]{0.13,0.29,0.53}{#1}}
\newcommand{\DecValTok}[1]{\textcolor[rgb]{0.00,0.00,0.81}{#1}}
\newcommand{\DocumentationTok}[1]{\textcolor[rgb]{0.56,0.35,0.01}{\textbf{\textit{#1}}}}
\newcommand{\ErrorTok}[1]{\textcolor[rgb]{0.64,0.00,0.00}{\textbf{#1}}}
\newcommand{\ExtensionTok}[1]{#1}
\newcommand{\FloatTok}[1]{\textcolor[rgb]{0.00,0.00,0.81}{#1}}
\newcommand{\FunctionTok}[1]{\textcolor[rgb]{0.00,0.00,0.00}{#1}}
\newcommand{\ImportTok}[1]{#1}
\newcommand{\InformationTok}[1]{\textcolor[rgb]{0.56,0.35,0.01}{\textbf{\textit{#1}}}}
\newcommand{\KeywordTok}[1]{\textcolor[rgb]{0.13,0.29,0.53}{\textbf{#1}}}
\newcommand{\NormalTok}[1]{#1}
\newcommand{\OperatorTok}[1]{\textcolor[rgb]{0.81,0.36,0.00}{\textbf{#1}}}
\newcommand{\OtherTok}[1]{\textcolor[rgb]{0.56,0.35,0.01}{#1}}
\newcommand{\PreprocessorTok}[1]{\textcolor[rgb]{0.56,0.35,0.01}{\textit{#1}}}
\newcommand{\RegionMarkerTok}[1]{#1}
\newcommand{\SpecialCharTok}[1]{\textcolor[rgb]{0.00,0.00,0.00}{#1}}
\newcommand{\SpecialStringTok}[1]{\textcolor[rgb]{0.31,0.60,0.02}{#1}}
\newcommand{\StringTok}[1]{\textcolor[rgb]{0.31,0.60,0.02}{#1}}
\newcommand{\VariableTok}[1]{\textcolor[rgb]{0.00,0.00,0.00}{#1}}
\newcommand{\VerbatimStringTok}[1]{\textcolor[rgb]{0.31,0.60,0.02}{#1}}
\newcommand{\WarningTok}[1]{\textcolor[rgb]{0.56,0.35,0.01}{\textbf{\textit{#1}}}}
\usepackage{graphicx}
\makeatletter
\def\maxwidth{\ifdim\Gin@nat@width>\linewidth\linewidth\else\Gin@nat@width\fi}
\def\maxheight{\ifdim\Gin@nat@height>\textheight\textheight\else\Gin@nat@height\fi}
\makeatother
% Scale images if necessary, so that they will not overflow the page
% margins by default, and it is still possible to overwrite the defaults
% using explicit options in \includegraphics[width, height, ...]{}
\setkeys{Gin}{width=\maxwidth,height=\maxheight,keepaspectratio}
% Set default figure placement to htbp
\makeatletter
\def\fps@figure{htbp}
\makeatother
\setlength{\emergencystretch}{3em} % prevent overfull lines
\providecommand{\tightlist}{%
  \setlength{\itemsep}{0pt}\setlength{\parskip}{0pt}}
\setcounter{secnumdepth}{-\maxdimen} % remove section numbering
\ifLuaTeX
  \usepackage{selnolig}  % disable illegal ligatures
\fi

\begin{document}

Ovdje neki uvod i razrada null vrijednosti ako napravimo

\begin{Shaded}
\begin{Highlighting}[]
\NormalTok{data }\OtherTok{\textless{}{-}} \FunctionTok{read.csv}\NormalTok{(}\StringTok{\textquotesingle{}./vgsales.csv\textquotesingle{}}\NormalTok{)}
\FunctionTok{library}\NormalTok{(tidyverse)}
\end{Highlighting}
\end{Shaded}

\begin{verbatim}
## Warning: package 'tidyverse' was built under R version 4.1.2
\end{verbatim}

\begin{verbatim}
## -- Attaching packages --------------------------------------- tidyverse 1.3.1 --
\end{verbatim}

\begin{verbatim}
## v ggplot2 3.3.5     v purrr   0.3.4
## v tibble  3.1.6     v dplyr   1.0.7
## v tidyr   1.1.4     v stringr 1.4.0
## v readr   2.1.1     v forcats 0.5.1
\end{verbatim}

\begin{verbatim}
## Warning: package 'ggplot2' was built under R version 4.1.2
\end{verbatim}

\begin{verbatim}
## Warning: package 'tibble' was built under R version 4.1.2
\end{verbatim}

\begin{verbatim}
## Warning: package 'tidyr' was built under R version 4.1.2
\end{verbatim}

\begin{verbatim}
## Warning: package 'readr' was built under R version 4.1.2
\end{verbatim}

\begin{verbatim}
## Warning: package 'purrr' was built under R version 4.1.2
\end{verbatim}

\begin{verbatim}
## Warning: package 'dplyr' was built under R version 4.1.2
\end{verbatim}

\begin{verbatim}
## Warning: package 'stringr' was built under R version 4.1.2
\end{verbatim}

\begin{verbatim}
## Warning: package 'forcats' was built under R version 4.1.2
\end{verbatim}

\begin{verbatim}
## -- Conflicts ------------------------------------------ tidyverse_conflicts() --
## x dplyr::filter() masks stats::filter()
## x dplyr::lag()    masks stats::lag()
\end{verbatim}

\#\#1. Pitanje: Jesu li u Japanu RPG igre značajno prodavanije od FPS
igara? Možete li pronaći neki žanr koji je značajno popularniji u Europi
nego u Japanu?

\#Jesu li u Japanu RPG igre značajno prodavanije od FPS igara?

\begin{Shaded}
\begin{Highlighting}[]
\CommentTok{\# Pogledajmo prvo podatke koje imam o Japanu prije same uspostave testa}
\CommentTok{\# Za usporedbu po žanrovima koristiti ćemo udio prodaje žanra u Japanu od ukupnih prodaja u Japanu.}

\NormalTok{JPGenreSales }\OtherTok{=}\NormalTok{ data[}\FunctionTok{c}\NormalTok{(}\DecValTok{5}\NormalTok{,}\DecValTok{9}\NormalTok{)]}

\NormalTok{JapanProfitByGenre }\OtherTok{=}\NormalTok{ JPGenreSales }\SpecialCharTok{\%\textgreater{}\%} \FunctionTok{group\_by}\NormalTok{(Genre) }\SpecialCharTok{\%\textgreater{}\%} 
                     \FunctionTok{summarise}\NormalTok{(}\AttributeTok{PercentageOfGenreInMarket =} \FunctionTok{sum}\NormalTok{(JP\_Sales) }\SpecialCharTok{/} \FunctionTok{sum}\NormalTok{(JPGenreSales}\SpecialCharTok{$}\NormalTok{JP\_Sales))}


\FunctionTok{barplot}\NormalTok{(JapanProfitByGenre}\SpecialCharTok{$}\NormalTok{PercentageOfGenreInMarket, }
        \AttributeTok{names.arg =}\NormalTok{ JapanProfitByGenre}\SpecialCharTok{$}\NormalTok{Genre, }
        \AttributeTok{las =} \DecValTok{2}\NormalTok{, }
        \AttributeTok{cex.names =} \FloatTok{0.8}\NormalTok{,}
        \AttributeTok{main =} \StringTok{"Popularnost žanrova u Japanu"}\NormalTok{,}
        \AttributeTok{ylab =} \StringTok{"Udio prodaje žanra od ukupne prodaje u Japana"}\NormalTok{)}
\end{Highlighting}
\end{Shaded}

\includegraphics{Test_files/figure-latex/unnamed-chunk-2-1.pdf}

Na prvi pogled izgleda kao da su RPG igre znatno dominantan žanr u
Japanu te svakako popularnije od FPS igri. Uvjerimo se u to
prikazivanjem podataka pomoću dodatnih grafova te zatim provjerimo
t-testom.

\begin{Shaded}
\begin{Highlighting}[]
\NormalTok{JP\_FPSData }\OtherTok{=}\NormalTok{ JPGenreSales[JPGenreSales}\SpecialCharTok{$}\NormalTok{Genre }\SpecialCharTok{==} \StringTok{"Shooter"} \SpecialCharTok{\&}\NormalTok{ JPGenreSales}\SpecialCharTok{$}\NormalTok{JP\_Sale }\SpecialCharTok{\textgreater{}} \DecValTok{0}\NormalTok{, ]}
\NormalTok{JP\_RPGData }\OtherTok{=}\NormalTok{ JPGenreSales[JPGenreSales}\SpecialCharTok{$}\NormalTok{Genre }\SpecialCharTok{==} \StringTok{"Role{-}Playing"} \SpecialCharTok{\&}\NormalTok{ JPGenreSales}\SpecialCharTok{$}\NormalTok{JP\_Sales }\SpecialCharTok{\textgreater{}} \DecValTok{0}\NormalTok{, ]}


\FunctionTok{boxplot}\NormalTok{(JP\_FPSData}\SpecialCharTok{$}\NormalTok{JP\_Sales, JP\_RPGData}\SpecialCharTok{$}\NormalTok{JP\_Sales,}
        \AttributeTok{names =} \FunctionTok{c}\NormalTok{(}\StringTok{"FPS sales"}\NormalTok{, }\StringTok{"RPG sales"}\NormalTok{),}
        \AttributeTok{main =} \StringTok{"Prodaja FPS i RPG igara u Japanu"}\NormalTok{)}
\end{Highlighting}
\end{Shaded}

\includegraphics{Test_files/figure-latex/unnamed-chunk-3-1.pdf}

\begin{Shaded}
\begin{Highlighting}[]
\NormalTok{b }\OtherTok{=} \FunctionTok{seq}\NormalTok{(}\FunctionTok{min}\NormalTok{(JP\_RPGData}\SpecialCharTok{$}\NormalTok{JP\_Sales) }\SpecialCharTok{{-}} \FloatTok{0.2}\NormalTok{,}\FunctionTok{max}\NormalTok{(JP\_RPGData}\SpecialCharTok{$}\NormalTok{JP\_Sales) }\SpecialCharTok{+} \FloatTok{0.2}\NormalTok{,}\FloatTok{0.2}\NormalTok{)}

\NormalTok{h1 }\OtherTok{=} \FunctionTok{hist}\NormalTok{(JP\_FPSData}\SpecialCharTok{$}\NormalTok{JP\_Sales, }\AttributeTok{breaks =}\NormalTok{ b, }\AttributeTok{plot =} \ConstantTok{FALSE}\NormalTok{)}
\NormalTok{h2 }\OtherTok{=} \FunctionTok{hist}\NormalTok{(JP\_RPGData}\SpecialCharTok{$}\NormalTok{JP\_Sales, }\AttributeTok{breaks =}\NormalTok{ b, }\AttributeTok{plot =} \ConstantTok{FALSE}\NormalTok{)}

\NormalTok{JP\_RPGFPSdata }\OtherTok{\textless{}{-}} \FunctionTok{t}\NormalTok{(}\FunctionTok{cbind}\NormalTok{(h1}\SpecialCharTok{$}\NormalTok{counts, h2}\SpecialCharTok{$}\NormalTok{counts))}

\FunctionTok{barplot}\NormalTok{(JP\_RPGFPSdata, }\AttributeTok{beside=}\ConstantTok{TRUE}\NormalTok{, }\AttributeTok{col =} \FunctionTok{c}\NormalTok{(}\StringTok{"red"}\NormalTok{, }\StringTok{"green"}\NormalTok{),}
        \AttributeTok{main =} \StringTok{"Prodaja FPS i RPG igara u Japanu"}\NormalTok{)}
\FunctionTok{legend}\NormalTok{(}\StringTok{"topright"}\NormalTok{, }\AttributeTok{fill =} \FunctionTok{c}\NormalTok{(}\StringTok{"red"}\NormalTok{, }\StringTok{"green"}\NormalTok{), }\FunctionTok{c}\NormalTok{(}\StringTok{"FPS igre"}\NormalTok{, }\StringTok{"RPG igre"}\NormalTok{))}
\end{Highlighting}
\end{Shaded}

\includegraphics{Test_files/figure-latex/unnamed-chunk-3-2.pdf}

Nažalost podatci sadrže ogroman broj niskoprodavanih igrica te time je
svaka imalo uspješna igrica ``outlier''. Posljedica je jako nečitljiv
box-and-whisker graf i gotovo nevidljivi outlieri na bar grafu. Zato
ćemo iz naših podataka izbaciti najnižih i najviših 10\% vrijednosti.

\begin{Shaded}
\begin{Highlighting}[]
\NormalTok{JP\_FPSq }\OtherTok{\textless{}{-}} \FunctionTok{quantile}\NormalTok{(JP\_FPSData}\SpecialCharTok{$}\NormalTok{JP\_Sales, }\AttributeTok{probs =} \FunctionTok{c}\NormalTok{(.}\DecValTok{1}\NormalTok{, .}\DecValTok{90}\NormalTok{))}
\NormalTok{JP\_RPGq }\OtherTok{\textless{}{-}} \FunctionTok{quantile}\NormalTok{(JP\_RPGData}\SpecialCharTok{$}\NormalTok{JP\_Sales, }\AttributeTok{probs =} \FunctionTok{c}\NormalTok{(.}\DecValTok{1}\NormalTok{, .}\DecValTok{90}\NormalTok{))}

\NormalTok{JP\_FPSDataTrimmed }\OtherTok{=}\NormalTok{ JPGenreSales[JPGenreSales}\SpecialCharTok{$}\NormalTok{Genre }\SpecialCharTok{==} \StringTok{"Shooter"} 
                              \SpecialCharTok{\&}\NormalTok{ JPGenreSales}\SpecialCharTok{$}\NormalTok{JP\_Sales }\SpecialCharTok{\textgreater{}=}\NormalTok{ JP\_FPSq[}\DecValTok{1}\NormalTok{] }\SpecialCharTok{\&}\NormalTok{ JPGenreSales}\SpecialCharTok{$}\NormalTok{JP\_Sales }\SpecialCharTok{\textless{}=}\NormalTok{ JP\_FPSq[}\DecValTok{2}\NormalTok{], ]}
\NormalTok{JP\_RPGDataTrimmed }\OtherTok{=}\NormalTok{ JPGenreSales[JPGenreSales}\SpecialCharTok{$}\NormalTok{Genre }\SpecialCharTok{==} \StringTok{"Role{-}Playing"} 
                              \SpecialCharTok{\&}\NormalTok{ JPGenreSales}\SpecialCharTok{$}\NormalTok{JP\_Sales }\SpecialCharTok{\textgreater{}=}\NormalTok{ JP\_RPGq[}\DecValTok{1}\NormalTok{] }\SpecialCharTok{\&}\NormalTok{ JPGenreSales}\SpecialCharTok{$}\NormalTok{JP\_Sales }\SpecialCharTok{\textless{}=}\NormalTok{ JP\_RPGq[}\DecValTok{2}\NormalTok{], ]}

\NormalTok{bTrimmed }\OtherTok{=} \FunctionTok{seq}\NormalTok{(}\FunctionTok{min}\NormalTok{(JP\_RPGDataTrimmed}\SpecialCharTok{$}\NormalTok{JP\_Sales) }\SpecialCharTok{{-}} \FloatTok{0.05}\NormalTok{,}\FunctionTok{max}\NormalTok{(JP\_RPGDataTrimmed}\SpecialCharTok{$}\NormalTok{JP\_Sales) }\SpecialCharTok{+} \FloatTok{0.05}\NormalTok{,}\FloatTok{0.05}\NormalTok{)}

\FunctionTok{boxplot}\NormalTok{(JP\_FPSDataTrimmed}\SpecialCharTok{$}\NormalTok{JP\_Sales, JP\_RPGDataTrimmed}\SpecialCharTok{$}\NormalTok{JP\_Sales,}
        \AttributeTok{names =} \FunctionTok{c}\NormalTok{(}\StringTok{"FPS sales"}\NormalTok{, }\StringTok{"RPG sales"}\NormalTok{),}
        \AttributeTok{main =} \StringTok{"Prodaja FPS i RPG igara u Japanu"}\NormalTok{)}
\end{Highlighting}
\end{Shaded}

\includegraphics{Test_files/figure-latex/unnamed-chunk-4-1.pdf}

\begin{Shaded}
\begin{Highlighting}[]
\NormalTok{h1Trimmed }\OtherTok{=} \FunctionTok{hist}\NormalTok{(JP\_FPSDataTrimmed}\SpecialCharTok{$}\NormalTok{JP\_Sales, }\AttributeTok{breaks =}\NormalTok{ bTrimmed, }\AttributeTok{plot =} \ConstantTok{FALSE}\NormalTok{)}
\NormalTok{h2Trimmed }\OtherTok{=} \FunctionTok{hist}\NormalTok{(JP\_RPGDataTrimmed}\SpecialCharTok{$}\NormalTok{JP\_Sales, }\AttributeTok{breaks =}\NormalTok{ bTrimmed, }\AttributeTok{plot =} \ConstantTok{FALSE}\NormalTok{)}

\NormalTok{JP\_RPGFPSdataTrimmed }\OtherTok{\textless{}{-}} \FunctionTok{t}\NormalTok{(}\FunctionTok{cbind}\NormalTok{(h1Trimmed}\SpecialCharTok{$}\NormalTok{counts, h2Trimmed}\SpecialCharTok{$}\NormalTok{counts))}

\FunctionTok{barplot}\NormalTok{(JP\_RPGFPSdataTrimmed, }\AttributeTok{beside=}\ConstantTok{TRUE}\NormalTok{, }\AttributeTok{col =} \FunctionTok{c}\NormalTok{(}\StringTok{"red"}\NormalTok{, }\StringTok{"green"}\NormalTok{),}
        \AttributeTok{main =} \StringTok{"Prodaja FPS i RPG igara u Japanu"}\NormalTok{,)}
\FunctionTok{legend}\NormalTok{(}\StringTok{"topright"}\NormalTok{, }\AttributeTok{fill =} \FunctionTok{c}\NormalTok{(}\StringTok{"red"}\NormalTok{, }\StringTok{"green"}\NormalTok{), }\FunctionTok{c}\NormalTok{(}\StringTok{"FPS igre"}\NormalTok{, }\StringTok{"RPG igre"}\NormalTok{))}
\end{Highlighting}
\end{Shaded}

\includegraphics{Test_files/figure-latex/unnamed-chunk-4-2.pdf}

Iz novih grafova možemo pretpostaviti nekoliko stvari: U Japanu izlazi
puno više novih RPG nego FPS igri. U prosjeku RPG igre se prodaju više
od FPS igri. RPG igre imaju puno više jako uspješnih outliera te su oni
i znatno uspješniji od FPS outliera. Potvrdimo sada pretpostavku
popularnosti t-testom.

\begin{Shaded}
\begin{Highlighting}[]
\CommentTok{\#Postavimo t{-}test na sljedeći način:}
\CommentTok{\#H0 {-} d = 0 {-}{-} srednje vrijednosti prodaje RPG i FPS igri su jednake}
\CommentTok{\#H1 {-} d \textgreater{} 0 {-}{-} srednje vrijednosti prodaje RPG i FPS igri nisu jednake }
\CommentTok{\#također koristiti ćemo alfa=0.01 da bismo dobili 99\% interval povjerenja}
\CommentTok{\#pretpostavljamo različite varijance prodaje igri unutar žanrova}

\FunctionTok{t.test}\NormalTok{(JP\_RPGData}\SpecialCharTok{$}\NormalTok{JP\_Sales, JP\_FPSData}\SpecialCharTok{$}\NormalTok{JP\_Sales, }
       \AttributeTok{alternative=}\StringTok{"greater"}\NormalTok{,}
       \AttributeTok{paired=}\ConstantTok{FALSE}\NormalTok{,}
       \AttributeTok{var.equal=}\ConstantTok{FALSE}\NormalTok{,}
       \AttributeTok{conf.level=}\FloatTok{0.99}\NormalTok{)}
\end{Highlighting}
\end{Shaded}

\begin{verbatim}
## 
##  Welch Two Sample t-test
## 
## data:  JP_RPGData$JP_Sales and JP_FPSData$JP_Sales
## t = 9.4633, df = 1301, p-value < 2.2e-16
## alternative hypothesis: true difference in means is greater than 0
## 99 percent confidence interval:
##  0.176123      Inf
## sample estimates:
## mean of x mean of y 
## 0.3348954 0.1012698
\end{verbatim}

\begin{Shaded}
\begin{Highlighting}[]
\FunctionTok{qt}\NormalTok{(}\AttributeTok{p=}\FloatTok{0.01}\NormalTok{, }\AttributeTok{df=}\DecValTok{1301}\NormalTok{, }\AttributeTok{lower.tail=}\ConstantTok{FALSE}\NormalTok{)}
\end{Highlighting}
\end{Shaded}

\begin{verbatim}
## [1] 2.329218
\end{verbatim}

\begin{Shaded}
\begin{Highlighting}[]
\FunctionTok{qt}\NormalTok{(}\AttributeTok{p=}\FloatTok{0.00001}\NormalTok{, }\AttributeTok{df=}\DecValTok{1301}\NormalTok{, }\AttributeTok{lower.tail=}\ConstantTok{FALSE}\NormalTok{) }\CommentTok{\#t{-}vrijednost potrebna za odbacivanje na 99.999\% intervalu povjerenja}
\end{Highlighting}
\end{Shaded}

\begin{verbatim}
## [1] 4.280668
\end{verbatim}

Iz rezultata možemo zaključiti sljedeće: Na temelju jako male
p-vrijednost znamo da je ovo jako ekstreman slučaj tj. jako je mala
šansa ovakvog uzorka u slučaju H0. P-vrijednost podupire hipotezu H1. Na
temelju jako velike t-vrijednosti koja se nalazi daleko iznad kritične
vrijednosti odbacivanja možemo odbaciti H0 u korist H1. Zaključujemo da
su RPG igre prodavanije od FPS igri.

\#Možete li pronaći neki žanr koji je značajno popularniji u Europi nego
u Japanu?

Prije nego krenemo na rješavanje samog pitanja moramo razriješiti
problem popularnosti. Kako je Europsko tržište znatno veće nego Japan
moramo odabrati model koji će dobro opisati ``popularnost'' žanrova
neovisno o populaciji potrošača. Model koji smo izabrali za grafički
prikaz problema jest udio potrošača žanra od ukupnih potrošača regije.
npr. popularnost RPG igara u Japanu odredit ćemo tako da broj prodaja
RPG igara Japana podijelimo s brojem prodaje svih žanrova Japana.

\begin{Shaded}
\begin{Highlighting}[]
\CommentTok{\#Prikažimo prvo podatke grafički bar grafovima}

\NormalTok{EUGenreSales }\OtherTok{=}\NormalTok{ data[}\FunctionTok{c}\NormalTok{(}\DecValTok{5}\NormalTok{,}\DecValTok{8}\NormalTok{)]}

\NormalTok{EuropeProfitByGenre }\OtherTok{=}\NormalTok{ EUGenreSales }\SpecialCharTok{\%\textgreater{}\%} \FunctionTok{group\_by}\NormalTok{(Genre) }\SpecialCharTok{\%\textgreater{}\%} 
                     \FunctionTok{summarise}\NormalTok{(}\AttributeTok{PercentageOfGenreInMarket =} \FunctionTok{sum}\NormalTok{(EU\_Sales) }\SpecialCharTok{/} \FunctionTok{sum}\NormalTok{(EUGenreSales}\SpecialCharTok{$}\NormalTok{EU\_Sales))}

\FunctionTok{par}\NormalTok{(}\AttributeTok{mfrow=}\FunctionTok{c}\NormalTok{(}\DecValTok{1}\NormalTok{,}\DecValTok{2}\NormalTok{))}

\FunctionTok{barplot}\NormalTok{(EuropeProfitByGenre}\SpecialCharTok{$}\NormalTok{PercentageOfGenreInMarket, }
        \AttributeTok{names.arg =}\NormalTok{ EuropeProfitByGenre}\SpecialCharTok{$}\NormalTok{Genre, }
        \AttributeTok{las =} \DecValTok{2}\NormalTok{, }
        \AttributeTok{cex.names =} \FloatTok{0.8}\NormalTok{,}
        \AttributeTok{main =} \StringTok{"Popularnost žanrova u Europi"}\NormalTok{,}
        \AttributeTok{ylab =} \StringTok{"Udio prodaje žanra od ukupne prodaje u Europi"}\NormalTok{)}

\FunctionTok{barplot}\NormalTok{(JapanProfitByGenre}\SpecialCharTok{$}\NormalTok{PercentageOfGenreInMarket, }
        \AttributeTok{names.arg =}\NormalTok{ JapanProfitByGenre}\SpecialCharTok{$}\NormalTok{Genre, }
        \AttributeTok{las =} \DecValTok{2}\NormalTok{, }
        \AttributeTok{cex.names =} \FloatTok{0.8}\NormalTok{,}
        \AttributeTok{main =} \StringTok{"Popularnost žanrova u Japanu"}\NormalTok{,}
        \AttributeTok{ylab =} \StringTok{"Udio prodaje žanra od ukupne prodaje u Japana"}\NormalTok{)}
\end{Highlighting}
\end{Shaded}

\includegraphics{Test_files/figure-latex/unnamed-chunk-6-1.pdf}

Možemo uočiti nekoliko žanrova za koje možemo pretpostaviti da su
popularniji u Europi nego u Japanu prema izabranom modelu. To su
žanrovi: Action, Racing, Shooter, Sports. Prikažimo dodatne pomoćne
grafove za žanr Shooter.

\begin{Shaded}
\begin{Highlighting}[]
\CommentTok{\#Odmah ćemo koristiti srezane podatke}
\NormalTok{EU\_FPSData }\OtherTok{=}\NormalTok{ EUGenreSales[EUGenreSales}\SpecialCharTok{$}\NormalTok{Genre }\SpecialCharTok{==} \StringTok{"Shooter"} \SpecialCharTok{\&}\NormalTok{ EUGenreSales}\SpecialCharTok{$}\NormalTok{EU\_Sales }\SpecialCharTok{\textgreater{}} \DecValTok{0}\NormalTok{, ]}

\NormalTok{EU\_FPSq }\OtherTok{\textless{}{-}} \FunctionTok{quantile}\NormalTok{(EU\_FPSData}\SpecialCharTok{$}\NormalTok{EU\_Sales, }\AttributeTok{probs =} \FunctionTok{c}\NormalTok{(.}\DecValTok{1}\NormalTok{, .}\DecValTok{90}\NormalTok{))}

\NormalTok{EU\_FPSDataTrimmed }\OtherTok{=}\NormalTok{ EUGenreSales[EUGenreSales}\SpecialCharTok{$}\NormalTok{Genre }\SpecialCharTok{==} \StringTok{"Shooter"} 
                        \SpecialCharTok{\&}\NormalTok{ EUGenreSales}\SpecialCharTok{$}\NormalTok{EU\_Sales }\SpecialCharTok{\textgreater{}=}\NormalTok{ EU\_FPSq[}\DecValTok{1}\NormalTok{] }\SpecialCharTok{\&}\NormalTok{ EUGenreSales}\SpecialCharTok{$}\NormalTok{EU\_Sales }\SpecialCharTok{\textless{}=}\NormalTok{ EU\_FPSq[}\DecValTok{2}\NormalTok{], ]}

\FunctionTok{boxplot}\NormalTok{(EU\_FPSDataTrimmed}\SpecialCharTok{$}\NormalTok{EU\_Sales, JP\_FPSDataTrimmed}\SpecialCharTok{$}\NormalTok{JP\_Sales,}
        \AttributeTok{names =} \FunctionTok{c}\NormalTok{(}\StringTok{"EU FPS sales"}\NormalTok{, }\StringTok{"Japan FPS sales"}\NormalTok{),}
        \AttributeTok{main =} \StringTok{"Prodaja FPS igri u EU i Japanu"}\NormalTok{)}
\end{Highlighting}
\end{Shaded}

\includegraphics{Test_files/figure-latex/unnamed-chunk-7-1.pdf}

\begin{Shaded}
\begin{Highlighting}[]
\NormalTok{bJPvEU }\OtherTok{=} \FunctionTok{seq}\NormalTok{(}\FunctionTok{min}\NormalTok{(JP\_FPSData}\SpecialCharTok{$}\NormalTok{JP\_Sales) }\SpecialCharTok{{-}} \FloatTok{0.05}\NormalTok{,}\FunctionTok{max}\NormalTok{(JP\_FPSData}\SpecialCharTok{$}\NormalTok{JP\_Sales)}\SpecialCharTok{{-}}\FloatTok{0.6}\NormalTok{,}\FloatTok{0.05}\NormalTok{)}

\NormalTok{h1 }\OtherTok{=} \FunctionTok{hist}\NormalTok{(JP\_FPSDataTrimmed}\SpecialCharTok{$}\NormalTok{JP\_Sales, }\AttributeTok{breaks =}\NormalTok{ bJPvEU, }\AttributeTok{plot =} \ConstantTok{FALSE}\NormalTok{)}
\NormalTok{h2 }\OtherTok{=} \FunctionTok{hist}\NormalTok{(EU\_FPSDataTrimmed}\SpecialCharTok{$}\NormalTok{EU\_Sales, }\AttributeTok{breaks =}\NormalTok{ bJPvEU, }\AttributeTok{plot =} \ConstantTok{FALSE}\NormalTok{)}

\NormalTok{JPEU\_FPSdataTrimmed }\OtherTok{\textless{}{-}} \FunctionTok{t}\NormalTok{(}\FunctionTok{cbind}\NormalTok{(h1}\SpecialCharTok{$}\NormalTok{counts, h2}\SpecialCharTok{$}\NormalTok{counts))}

\FunctionTok{barplot}\NormalTok{(JPEU\_FPSdataTrimmed, }\AttributeTok{beside=}\ConstantTok{TRUE}\NormalTok{, }\AttributeTok{col =} \FunctionTok{c}\NormalTok{(}\StringTok{"red"}\NormalTok{, }\StringTok{"green"}\NormalTok{),}
        \AttributeTok{main =} \StringTok{"Prodaja FPS igri u EU i Japanu"}\NormalTok{)}
\FunctionTok{legend}\NormalTok{(}\StringTok{"topright"}\NormalTok{, }\AttributeTok{fill =} \FunctionTok{c}\NormalTok{(}\StringTok{"red"}\NormalTok{, }\StringTok{"green"}\NormalTok{), }\FunctionTok{c}\NormalTok{(}\StringTok{"FPS prodaja u Japanu"}\NormalTok{, }\StringTok{"FPS prodaja u EU"}\NormalTok{))}
\end{Highlighting}
\end{Shaded}

\includegraphics{Test_files/figure-latex/unnamed-chunk-7-2.pdf}

Prikazani grafovi potvrđuju da su FPS igre popularnije u Europi nego
Japanu i po broju naslova i po prodajama. No potvrdimo to t-testom.

\begin{Shaded}
\begin{Highlighting}[]
\CommentTok{\#Kako bismo dobili podatke koji bolje prikazuju popularnost unutar same regije, vrijednosti prodaje FPS igara unutar regije podijelit ćemo s prosječnim brojem prodaje igara unutar te regije}

\NormalTok{NonBisaedEUFPSData }\OtherTok{=}\NormalTok{ EU\_FPSData}\SpecialCharTok{$}\NormalTok{EU\_Sales }\SpecialCharTok{/} \FunctionTok{mean}\NormalTok{(data}\SpecialCharTok{$}\NormalTok{EU\_Sales)}
\NormalTok{NonBiasedJPFPSDATA }\OtherTok{=}\NormalTok{ JP\_FPSData}\SpecialCharTok{$}\NormalTok{JP\_Sales }\SpecialCharTok{/} \FunctionTok{mean}\NormalTok{(data}\SpecialCharTok{$}\NormalTok{JP\_Sales)}

\CommentTok{\#Postavimo t{-}test na sljedeći način:}
\CommentTok{\#H0 {-} d = 0 {-}{-} srednje vrijednosti prodaje FPS igri unutar EU i Japana su jednake}
\CommentTok{\#H1 {-} d \textgreater{} 0 {-}{-} srednje vrijednosti prodaje FPS igri unutar EU i Japana nisu jednake }
\CommentTok{\#također koristiti ćemo alfa=0.01 da bismo dobili 99\% interval povjerenja}
\CommentTok{\#pretpostavljamo različite varijance prodaje igri unutar žanrova}

\FunctionTok{t.test}\NormalTok{(NonBisaedEUFPSData, NonBiasedJPFPSDATA, }
       \AttributeTok{alternative=}\StringTok{"greater"}\NormalTok{,}
       \AttributeTok{paired=}\ConstantTok{FALSE}\NormalTok{,}
       \AttributeTok{var.equal=}\ConstantTok{FALSE}\NormalTok{,}
       \AttributeTok{conf.level=}\FloatTok{0.99}\NormalTok{)}
\end{Highlighting}
\end{Shaded}

\begin{verbatim}
## 
##  Welch Two Sample t-test
## 
## data:  NonBisaedEUFPSData and NonBiasedJPFPSDATA
## t = 4.0421, df = 1208.4, p-value = 2.816e-05
## alternative hypothesis: true difference in means is greater than 0
## 99 percent confidence interval:
##  0.2856255       Inf
## sample estimates:
## mean of x mean of y 
##  1.976083  1.301976
\end{verbatim}

\begin{Shaded}
\begin{Highlighting}[]
\FunctionTok{qt}\NormalTok{(}\AttributeTok{p=}\FloatTok{0.0001}\NormalTok{, }\AttributeTok{df=}\DecValTok{1301}\NormalTok{, }\AttributeTok{lower.tail=}\ConstantTok{FALSE}\NormalTok{)}
\end{Highlighting}
\end{Shaded}

\begin{verbatim}
## [1] 3.729643
\end{verbatim}

\begin{Shaded}
\begin{Highlighting}[]
\FunctionTok{qt}\NormalTok{(}\AttributeTok{p=}\FloatTok{0.00001}\NormalTok{, }\AttributeTok{df=}\DecValTok{1301}\NormalTok{, }\AttributeTok{lower.tail=}\ConstantTok{FALSE}\NormalTok{) }\CommentTok{\#t{-}vrijednost potrebna za odbacivanje na 99.999\% intervalu povjerenja}
\end{Highlighting}
\end{Shaded}

\begin{verbatim}
## [1] 4.280668
\end{verbatim}

Iz rezultata možemo zaključiti sljedeće: Na temelju jako male
p-vrijednost znamo da je ovo jako ekstreman slučaj tj. jako je mala
šansa ovakvog uzorka u slučaju H0. P-vrijednost podupire hipotezu H1. Na
temelju jako velike t-vrijednosti (nalazi se između intervala povjerenja
99.99\% i 99.999\%) koja se nalazi iznad kritične vrijednosti
odbacivanja možemo odbaciti H0 u korist H1. Zaključujemo da su FPS igre
prodavanije u EU nego Japanu.

\#\#2. Pitanje: Promatrajući prodaju u Sjevernoj Americi, jesu li neki
žanrovi značajno popularniji?

Gledamo prvo podatke koje imamo o Sjevernoj Americi prije testa. Za
usporedbu po žanrovima koristiti ćemo udio prodaje žanrova u Americi od
ukupne prodaje u Americi.

\begin{Shaded}
\begin{Highlighting}[]
\NormalTok{NAGenreSales }\OtherTok{=}\NormalTok{ data[}\FunctionTok{c}\NormalTok{(}\DecValTok{5}\NormalTok{,}\DecValTok{7}\NormalTok{)]}
\NormalTok{NAProfitByGenre }\OtherTok{=}\NormalTok{ NAGenreSales }\SpecialCharTok{\%\textgreater{}\%} \FunctionTok{group\_by}\NormalTok{(Genre) }\SpecialCharTok{\%\textgreater{}\%} 
                     \FunctionTok{summarise}\NormalTok{(}\AttributeTok{PercentageOfGenreInMarket =} \FunctionTok{sum}\NormalTok{(NA\_Sales) }\SpecialCharTok{/} \FunctionTok{sum}\NormalTok{(NAGenreSales}\SpecialCharTok{$}\NormalTok{NA\_Sales))}
\FunctionTok{barplot}\NormalTok{(NAProfitByGenre}\SpecialCharTok{$}\NormalTok{PercentageOfGenreInMarket, }
        \AttributeTok{names.arg =}\NormalTok{ NAProfitByGenre}\SpecialCharTok{$}\NormalTok{Genre, }
        \AttributeTok{las =} \DecValTok{2}\NormalTok{, }
        \AttributeTok{cex.names =} \FloatTok{0.8}\NormalTok{,}
        \AttributeTok{main =} \StringTok{"Popularnost zanrova u Americi"}\NormalTok{,}
        \AttributeTok{ylab =} \StringTok{"Udio prodaje zanra od ukupne prodaje u Sjevernoj Americi"}\NormalTok{)}
\end{Highlighting}
\end{Shaded}

\includegraphics{Test_files/figure-latex/unnamed-chunk-9-1.pdf}

Odokativno vidimo da su prodavaniji žanrovi Action, Shooter i Sports, te
da se žanrovi Strategy,Adventure i Puzzle slabije prodaju, tj. možemo
pretpostaviti da varijance neće biti iste kod različitih žanrova.

Da bi smo odgovorili na ovo pitanje u potpunosti koristiti ćemo ANOVA
test, tj. metodu s kojom testiramo sredine više populacija, u našem
slučaju žanrova. Jedan od glavnih ciljeva analize varijance je
ustanoviti jesu li upravo te razlike između grupa samo posljedica
slučajnosti ili je statistički značajna. U našem slučaju koristimo
jednofaktorsku varijantu jer smatramo da popularnost proizlazi iz
prodanosti određenih žanrova.

Pretpostavke ANOVA-e su:

\begin{itemize}
\tightlist
\item
  nezavisnost pojedinih podataka u uzorcima,
\item
  normalna razdioba podataka,
\item
  homogenost varijanci među populacijama.
\end{itemize}

Analizom varijance testiramo: \[ \begin{aligned}
  H_0 & : \mu_1 = \mu_2 = \ldots = \mu_k \\
  H_1 & : \neg H_0.
\end{aligned} \]

Provjeru normalnost ćemo napraviti prikazom histograma svakog žanra.
Gdje je žanr varijabla koja određuje grupu, a prodaja je zavisna
varijabla.

\begin{Shaded}
\begin{Highlighting}[]
\FunctionTok{require}\NormalTok{(nortest)}
\end{Highlighting}
\end{Shaded}

\begin{verbatim}
## Loading required package: nortest
\end{verbatim}

\begin{Shaded}
\begin{Highlighting}[]
\CommentTok{\# Iz sljedećih histograma vidimo da podaci nisu normalno distribuirani}

\FunctionTok{hist}\NormalTok{(data}\SpecialCharTok{$}\NormalTok{NA\_Sales[data}\SpecialCharTok{$}\NormalTok{Genre }\SpecialCharTok{==} \StringTok{\textquotesingle{}Action\textquotesingle{}}\NormalTok{],}\AttributeTok{main=}\StringTok{\textquotesingle{}Broj prodanih primjeraka Action žanra iz ne transformiranih podataka\textquotesingle{}}\NormalTok{, }\AttributeTok{xlab=}\StringTok{\textquotesingle{}Value\textquotesingle{}}\NormalTok{,}\AttributeTok{ylab=}\StringTok{\textquotesingle{}Frequency\textquotesingle{}}\NormalTok{)}
\end{Highlighting}
\end{Shaded}

\includegraphics{Test_files/figure-latex/test pretpostavki - normalnost-1.pdf}

\begin{Shaded}
\begin{Highlighting}[]
\FunctionTok{hist}\NormalTok{(data}\SpecialCharTok{$}\NormalTok{NA\_Sales[data}\SpecialCharTok{$}\NormalTok{Genre }\SpecialCharTok{==} \StringTok{\textquotesingle{}Adventure\textquotesingle{}}\NormalTok{],}\AttributeTok{main=}\StringTok{\textquotesingle{}Broj prodanih primjeraka Adventure žanra iz ne transformiranih podataka\textquotesingle{}}\NormalTok{, }\AttributeTok{xlab=}\StringTok{\textquotesingle{}Value\textquotesingle{}}\NormalTok{,}\AttributeTok{ylab=}\StringTok{\textquotesingle{}Frequency\textquotesingle{}}\NormalTok{)}
\end{Highlighting}
\end{Shaded}

\includegraphics{Test_files/figure-latex/test pretpostavki - normalnost-2.pdf}

\includegraphics{Test_files/figure-latex/unnamed-chunk-10-1.pdf}
\includegraphics{Test_files/figure-latex/unnamed-chunk-10-2.pdf}
\includegraphics{Test_files/figure-latex/unnamed-chunk-10-3.pdf}
\includegraphics{Test_files/figure-latex/unnamed-chunk-10-4.pdf}
\includegraphics{Test_files/figure-latex/unnamed-chunk-10-5.pdf}
\includegraphics{Test_files/figure-latex/unnamed-chunk-10-6.pdf}
\includegraphics{Test_files/figure-latex/unnamed-chunk-10-7.pdf}
\includegraphics{Test_files/figure-latex/unnamed-chunk-10-8.pdf}
\includegraphics{Test_files/figure-latex/unnamed-chunk-10-9.pdf}

Pošto podatci nisu u normalnoj distribuciji, koristeći log
transformaciju približavamo podatke normalnoj distribuciji. Radimo novi
stupac gdje ćemo imati vrijednosti prodanih primjeraka povećanih 100
puta da nemamo vrijednosti između 0 i 1, jer nam onda logaritamska
transformacija ima samo pozitivne vrijednosti

\begin{Shaded}
\begin{Highlighting}[]
\NormalTok{trimmedData }\OtherTok{\textless{}{-}}\NormalTok{ data[data}\SpecialCharTok{$}\NormalTok{NA\_Sales }\SpecialCharTok{!=} \DecValTok{0} \SpecialCharTok{\&}\NormalTok{ data}\SpecialCharTok{$}\NormalTok{Year }\SpecialCharTok{!=} \StringTok{"N/A"} \SpecialCharTok{\&}\NormalTok{ data}\SpecialCharTok{$}\NormalTok{Year }\SpecialCharTok{!=} \StringTok{"2020"}\NormalTok{,]}
\NormalTok{trimmedData}\SpecialCharTok{$}\NormalTok{NA\_Sales\_adjusted }\OtherTok{\textless{}{-}} \FunctionTok{log}\NormalTok{(trimmedData}\SpecialCharTok{$}\NormalTok{NA\_Sales }\SpecialCharTok{*} \DecValTok{100}\NormalTok{)}


\FunctionTok{hist}\NormalTok{(trimmedData}\SpecialCharTok{$}\NormalTok{NA\_Sales\_adjusted[trimmedData}\SpecialCharTok{$}\NormalTok{Genre}\SpecialCharTok{==}\StringTok{\textquotesingle{}Action\textquotesingle{}}\NormalTok{],}\AttributeTok{main=}\StringTok{\textquotesingle{}Broj prodanih primjeraka Action žanra\textquotesingle{}}\NormalTok{,}\AttributeTok{xlab=}\StringTok{\textquotesingle{}Value\textquotesingle{}}\NormalTok{,}\AttributeTok{ylab=}\StringTok{\textquotesingle{}Frequency\textquotesingle{}}\NormalTok{, }\AttributeTok{breaks=}\DecValTok{20}\NormalTok{)}
\end{Highlighting}
\end{Shaded}

\includegraphics{Test_files/figure-latex/histogrami normalnosti-1.pdf}

\begin{Shaded}
\begin{Highlighting}[]
\FunctionTok{hist}\NormalTok{(trimmedData}\SpecialCharTok{$}\NormalTok{NA\_Sales\_adjusted[trimmedData}\SpecialCharTok{$}\NormalTok{Genre}\SpecialCharTok{==}\StringTok{\textquotesingle{}Racing\textquotesingle{}}\NormalTok{],}\AttributeTok{main=}\StringTok{\textquotesingle{}Broj prodanih primjeraka Racing žanra\textquotesingle{}}\NormalTok{,}\AttributeTok{xlab=}\StringTok{\textquotesingle{}Value\textquotesingle{}}\NormalTok{,}\AttributeTok{ylab=}\StringTok{\textquotesingle{}Frequency\textquotesingle{}}\NormalTok{, }\AttributeTok{breaks=}\DecValTok{20}\NormalTok{)}
\end{Highlighting}
\end{Shaded}

\includegraphics{Test_files/figure-latex/histogrami normalnosti-2.pdf}

\begin{Shaded}
\begin{Highlighting}[]
\FunctionTok{qqnorm}\NormalTok{(trimmedData}\SpecialCharTok{$}\NormalTok{NA\_Sales\_adjusted[trimmedData}\SpecialCharTok{$}\NormalTok{Genre}\SpecialCharTok{==}\StringTok{\textquotesingle{}Action\textquotesingle{}}\NormalTok{], }\AttributeTok{pch =} \DecValTok{1}\NormalTok{, }\AttributeTok{frame =} \ConstantTok{FALSE}\NormalTok{,}\AttributeTok{main=}\StringTok{\textquotesingle{}Action\textquotesingle{}}\NormalTok{)}
\FunctionTok{qqline}\NormalTok{(trimmedData}\SpecialCharTok{$}\NormalTok{NA\_Sales\_adjusted[trimmedData}\SpecialCharTok{$}\NormalTok{Genre}\SpecialCharTok{==}\StringTok{\textquotesingle{}Action\textquotesingle{}}\NormalTok{], }\AttributeTok{col =} \StringTok{"steelblue"}\NormalTok{, }\AttributeTok{lwd =} \DecValTok{2}\NormalTok{)}
\end{Highlighting}
\end{Shaded}

\includegraphics{Test_files/figure-latex/histogrami normalnosti-3.pdf}

\begin{Shaded}
\begin{Highlighting}[]
\FunctionTok{qqnorm}\NormalTok{(trimmedData}\SpecialCharTok{$}\NormalTok{NA\_Sales\_adjusted[trimmedData}\SpecialCharTok{$}\NormalTok{Genre}\SpecialCharTok{==}\StringTok{\textquotesingle{}Racing\textquotesingle{}}\NormalTok{], }\AttributeTok{pch =} \DecValTok{1}\NormalTok{, }\AttributeTok{frame =} \ConstantTok{FALSE}\NormalTok{,}\AttributeTok{main=}\StringTok{\textquotesingle{}Racing\textquotesingle{}}\NormalTok{)}
\FunctionTok{qqline}\NormalTok{(trimmedData}\SpecialCharTok{$}\NormalTok{NA\_Sales\_adjusted[trimmedData}\SpecialCharTok{$}\NormalTok{Genre}\SpecialCharTok{==}\StringTok{\textquotesingle{}Racing\textquotesingle{}}\NormalTok{], }\AttributeTok{col =} \StringTok{"steelblue"}\NormalTok{, }\AttributeTok{lwd =} \DecValTok{2}\NormalTok{)}
\end{Highlighting}
\end{Shaded}

\includegraphics{Test_files/figure-latex/histogrami normalnosti-4.pdf}

\includegraphics{Test_files/figure-latex/unnamed-chunk-11-1.pdf}
\includegraphics{Test_files/figure-latex/unnamed-chunk-11-2.pdf}
\includegraphics{Test_files/figure-latex/unnamed-chunk-11-3.pdf}
\includegraphics{Test_files/figure-latex/unnamed-chunk-11-4.pdf}
\includegraphics{Test_files/figure-latex/unnamed-chunk-11-5.pdf}
\includegraphics{Test_files/figure-latex/unnamed-chunk-11-6.pdf}
\includegraphics{Test_files/figure-latex/unnamed-chunk-11-7.pdf}
\includegraphics{Test_files/figure-latex/unnamed-chunk-11-8.pdf}
\includegraphics{Test_files/figure-latex/unnamed-chunk-11-9.pdf}
\includegraphics{Test_files/figure-latex/unnamed-chunk-11-10.pdf}
\includegraphics{Test_files/figure-latex/unnamed-chunk-11-11.pdf}
\includegraphics{Test_files/figure-latex/unnamed-chunk-11-12.pdf}
\includegraphics{Test_files/figure-latex/unnamed-chunk-11-13.pdf}
\includegraphics{Test_files/figure-latex/unnamed-chunk-11-14.pdf}
\includegraphics{Test_files/figure-latex/unnamed-chunk-11-15.pdf}
\includegraphics{Test_files/figure-latex/unnamed-chunk-11-16.pdf}
\includegraphics{Test_files/figure-latex/unnamed-chunk-11-17.pdf}
\includegraphics{Test_files/figure-latex/unnamed-chunk-11-18.pdf}

Provodimo Bartlettov test homogenosti varijanci među žanrovima i
nacrtajmo box plotove transformiranih podataka.

\begin{Shaded}
\begin{Highlighting}[]
\FunctionTok{bartlett.test}\NormalTok{(trimmedData}\SpecialCharTok{$}\NormalTok{NA\_Sales\_adjusted }\SpecialCharTok{\textasciitilde{}}\NormalTok{ trimmedData}\SpecialCharTok{$}\NormalTok{Genre)}
\end{Highlighting}
\end{Shaded}

\begin{verbatim}
## 
##  Bartlett test of homogeneity of variances
## 
## data:  trimmedData$NA_Sales_adjusted by trimmedData$Genre
## Bartlett's K-squared = 98.682, df = 11, p-value = 3.256e-16
\end{verbatim}

\begin{Shaded}
\begin{Highlighting}[]
\CommentTok{\# Graficki prikaz podataka}
\FunctionTok{boxplot}\NormalTok{(trimmedData}\SpecialCharTok{$}\NormalTok{NA\_Sales\_adjusted }\SpecialCharTok{\textasciitilde{}}\NormalTok{ trimmedData}\SpecialCharTok{$}\NormalTok{Genre, }\AttributeTok{cex.axis =} \DecValTok{1}\NormalTok{, }\AttributeTok{las=}\DecValTok{2}\NormalTok{, }\AttributeTok{xlab=}\StringTok{""}\NormalTok{, }\AttributeTok{ylab=}\StringTok{"Log transformirani podaci prodaje primjeraka u Americi"}\NormalTok{)}
\end{Highlighting}
\end{Shaded}

\includegraphics{Test_files/figure-latex/test pretpostavki - homogenost varijanci-1.pdf}
Zbog jako male p vrijednosti nam zapravo Bartlettov test sam sugerira da
su im varijance različite, no provedimo ANOVA test.

\begin{Shaded}
\begin{Highlighting}[]
\CommentTok{\# ANOVA Test}
\NormalTok{a }\OtherTok{=} \FunctionTok{aov}\NormalTok{(trimmedData}\SpecialCharTok{$}\NormalTok{NA\_Sales\_adjusted }\SpecialCharTok{\textasciitilde{}}\NormalTok{ trimmedData}\SpecialCharTok{$}\NormalTok{Genre)}
\FunctionTok{summary}\NormalTok{(a)}
\end{Highlighting}
\end{Shaded}

\begin{verbatim}
##                      Df Sum Sq Mean Sq F value Pr(>F)    
## trimmedData$Genre    11    487   44.31   26.63 <2e-16 ***
## Residuals         11875  19761    1.66                   
## ---
## Signif. codes:  0 '***' 0.001 '**' 0.01 '*' 0.05 '.' 0.1 ' ' 1
\end{verbatim}

\begin{Shaded}
\begin{Highlighting}[]
\CommentTok{\# Linearni model}
\NormalTok{model }\OtherTok{=} \FunctionTok{lm}\NormalTok{(trimmedData}\SpecialCharTok{$}\NormalTok{NA\_Sales\_adjusted }\SpecialCharTok{\textasciitilde{}}\NormalTok{ trimmedData}\SpecialCharTok{$}\NormalTok{Genre, }\AttributeTok{data =}\NormalTok{ trimmedData)}
\FunctionTok{summary}\NormalTok{(model)}
\end{Highlighting}
\end{Shaded}

\begin{verbatim}
## 
## Call:
## lm(formula = trimmedData$NA_Sales_adjusted ~ trimmedData$Genre, 
##     data = trimmedData)
## 
## Residuals:
##     Min      1Q  Median      3Q     Max 
## -2.9260 -0.8466 -0.0465  0.8406  5.5709 
## 
## Coefficients:
##                               Estimate Std. Error t value Pr(>|t|)    
## (Intercept)                    2.74295    0.02567 106.867  < 2e-16 ***
## trimmedData$GenreAdventure    -0.51320    0.06121  -8.385  < 2e-16 ***
## trimmedData$GenreFighting      0.09839    0.05919   1.662 0.096470 .  
## trimmedData$GenreMisc         -0.04369    0.04540  -0.962 0.335863    
## trimmedData$GenrePlatform      0.18309    0.05254   3.485 0.000494 ***
## trimmedData$GenrePuzzle       -0.56455    0.06670  -8.463  < 2e-16 ***
## trimmedData$GenreRacing       -0.19399    0.04678  -4.147 3.39e-05 ***
## trimmedData$GenreRole-Playing -0.05737    0.05109  -1.123 0.261409    
## trimmedData$GenreShooter       0.08503    0.04627   1.838 0.066123 .  
## trimmedData$GenreSimulation   -0.09751    0.05843  -1.669 0.095176 .  
## trimmedData$GenreSports        0.06391    0.03968   1.611 0.107307    
## trimmedData$GenreStrategy     -0.66485    0.07511  -8.852  < 2e-16 ***
## ---
## Signif. codes:  0 '***' 0.001 '**' 0.01 '*' 0.05 '.' 0.1 ' ' 1
## 
## Residual standard error: 1.29 on 11875 degrees of freedom
## Multiple R-squared:  0.02407,    Adjusted R-squared:  0.02317 
## F-statistic: 26.63 on 11 and 11875 DF,  p-value: < 2.2e-16
\end{verbatim}

\begin{Shaded}
\begin{Highlighting}[]
\FunctionTok{anova}\NormalTok{(model)}
\end{Highlighting}
\end{Shaded}

\begin{verbatim}
## Analysis of Variance Table
## 
## Response: trimmedData$NA_Sales_adjusted
##                      Df  Sum Sq Mean Sq F value    Pr(>F)    
## trimmedData$Genre    11   487.4  44.313  26.628 < 2.2e-16 ***
## Residuals         11875 19761.5   1.664                      
## ---
## Signif. codes:  0 '***' 0.001 '**' 0.01 '*' 0.05 '.' 0.1 ' ' 1
\end{verbatim}

Nakon provedenog ANOVA testa, zbog premalog p-valuea, zaključujemo da su
varijance različite, tj neki žanrovi su značajno popularniji, kao što
smo i očekivali.

No što ako provjeravamo točno one žanrove koje ``odokativno'' vidimo da
bi mogli imati istu popularnost. Provjerimo žanrove Strategy, Puzzle i
Adventure, je li ima značajno popularnijih? (znamo da su podaci normalno
distribuirani iz histograma provedenih prethodno)

\begin{Shaded}
\begin{Highlighting}[]
\NormalTok{adjustedData }\OtherTok{\textless{}{-}}\NormalTok{ data[(data}\SpecialCharTok{$}\NormalTok{Genre }\SpecialCharTok{==} \StringTok{"Strategy"} \SpecialCharTok{|}\NormalTok{ data}\SpecialCharTok{$}\NormalTok{Genre }\SpecialCharTok{==} \StringTok{"Puzzle"} \SpecialCharTok{|}\NormalTok{ data}\SpecialCharTok{$}\NormalTok{Genre }\SpecialCharTok{==} \StringTok{"Adventure"}\NormalTok{) }\SpecialCharTok{\&}\NormalTok{ data}\SpecialCharTok{$}\NormalTok{NA\_Sales }\SpecialCharTok{!=} \DecValTok{0}\NormalTok{,]}
\NormalTok{adjustedData}\SpecialCharTok{$}\NormalTok{NA\_Sales\_adjusted }\OtherTok{\textless{}{-}} \FunctionTok{log}\NormalTok{(adjustedData}\SpecialCharTok{$}\NormalTok{NA\_Sales }\SpecialCharTok{*} \DecValTok{100}\NormalTok{)}

\FunctionTok{bartlett.test}\NormalTok{(adjustedData}\SpecialCharTok{$}\NormalTok{NA\_Sales\_adjusted }\SpecialCharTok{\textasciitilde{}}\NormalTok{ adjustedData}\SpecialCharTok{$}\NormalTok{Genre)}
\end{Highlighting}
\end{Shaded}

\begin{verbatim}
## 
##  Bartlett test of homogeneity of variances
## 
## data:  adjustedData$NA_Sales_adjusted by adjustedData$Genre
## Bartlett's K-squared = 11.875, df = 2, p-value = 0.002639
\end{verbatim}

\begin{Shaded}
\begin{Highlighting}[]
\CommentTok{\# ANOVA Test}
\NormalTok{a }\OtherTok{=} \FunctionTok{aov}\NormalTok{(adjustedData}\SpecialCharTok{$}\NormalTok{NA\_Sales\_adjusted }\SpecialCharTok{\textasciitilde{}}\NormalTok{ adjustedData}\SpecialCharTok{$}\NormalTok{Genre)}
\FunctionTok{summary}\NormalTok{(a)}
\end{Highlighting}
\end{Shaded}

\begin{verbatim}
##                      Df Sum Sq Mean Sq F value Pr(>F)
## adjustedData$Genre    2      5   2.520    1.61    0.2
## Residuals          1330   2082   1.565
\end{verbatim}

Nakon provedbe ANOVA testa vidimo da je p vrijednost 0.2 što znači da
prihvaćamo H0 hipotezu da su varijance jednake, tj. nema značajno
popularnijih žanrova od odabranih.

Idemo provjeriti i prethodno spomenute žanrove Action, Shooter i Sports.

\begin{Shaded}
\begin{Highlighting}[]
\NormalTok{adjustedData }\OtherTok{\textless{}{-}}\NormalTok{ data[(data}\SpecialCharTok{$}\NormalTok{Genre }\SpecialCharTok{==} \StringTok{"Action"} \SpecialCharTok{|}\NormalTok{ data}\SpecialCharTok{$}\NormalTok{Genre }\SpecialCharTok{==} \StringTok{"Shooter"} \SpecialCharTok{|}\NormalTok{ data}\SpecialCharTok{$}\NormalTok{Genre }\SpecialCharTok{==} \StringTok{"Sports"}\NormalTok{) }\SpecialCharTok{\&}\NormalTok{ data}\SpecialCharTok{$}\NormalTok{NA\_Sales }\SpecialCharTok{!=} \DecValTok{0}\NormalTok{,]}
\NormalTok{adjustedData}\SpecialCharTok{$}\NormalTok{NA\_Sales\_adjusted }\OtherTok{\textless{}{-}} \FunctionTok{log}\NormalTok{(adjustedData}\SpecialCharTok{$}\NormalTok{NA\_Sales }\SpecialCharTok{*} \DecValTok{100}\NormalTok{)}

\FunctionTok{bartlett.test}\NormalTok{(adjustedData}\SpecialCharTok{$}\NormalTok{NA\_Sales\_adjusted }\SpecialCharTok{\textasciitilde{}}\NormalTok{ adjustedData}\SpecialCharTok{$}\NormalTok{Genre)}
\end{Highlighting}
\end{Shaded}

\begin{verbatim}
## 
##  Bartlett test of homogeneity of variances
## 
## data:  adjustedData$NA_Sales_adjusted by adjustedData$Genre
## Bartlett's K-squared = 56.267, df = 2, p-value = 6.049e-13
\end{verbatim}

\begin{Shaded}
\begin{Highlighting}[]
\CommentTok{\# ANOVA Test}
\NormalTok{a }\OtherTok{=} \FunctionTok{aov}\NormalTok{(adjustedData}\SpecialCharTok{$}\NormalTok{NA\_Sales\_adjusted }\SpecialCharTok{\textasciitilde{}}\NormalTok{ adjustedData}\SpecialCharTok{$}\NormalTok{Genre)}
\FunctionTok{summary}\NormalTok{(a)}
\end{Highlighting}
\end{Shaded}

\begin{verbatim}
##                      Df Sum Sq Mean Sq F value Pr(>F)
## adjustedData$Genre    2      6   3.181   1.918  0.147
## Residuals          5579   9251   1.658
\end{verbatim}

Dolazimo do istog zaključka kao i u prethodnom testu, tj. žanrovi
Action, Shooter i Sports su slične popularnosti.

\#\#3. Pitanje: Možemo li temeljem danih varijabli predvidjeti prodaju
neke videoigre?

Kako bi znali predvidjeti prodaju neke videoigre, možemo ispitati
različite varijable koje bi mogle znatno utjecati na broj prodanih
primjeraka u svijetu:

\begin{itemize}
\tightlist
\item
  Platforma
\item
  Žanr
\end{itemize}

Također, koristiti ćemo samo podatke zadnjih 5 godina u zadanim podacima
(\textgreater= 2012) kako bi vidjeli modernije trendove. Podatke koji
nemaju upisanu godinu cemo izbaciti.Pošto imamo samo 4 videoigre u
razdoblju (2017\textless=x\textless=2020) njih ćemo izbaciti.

\begin{Shaded}
\begin{Highlighting}[]
\NormalTok{PredictionData }\OtherTok{=}\NormalTok{ data[}\FunctionTok{c}\NormalTok{(}\DecValTok{3}\NormalTok{,}\DecValTok{4}\NormalTok{,}\DecValTok{5}\NormalTok{,}\DecValTok{6}\NormalTok{,}\DecValTok{11}\NormalTok{)]}

\NormalTok{Last5YearData }\OtherTok{=}\NormalTok{ PredictionData[PredictionData}\SpecialCharTok{$}\NormalTok{Year }\SpecialCharTok{\textgreater{}=} \DecValTok{2012}\NormalTok{,]}

\NormalTok{Last5YearData }\OtherTok{=}\NormalTok{ Last5YearData[Last5YearData}\SpecialCharTok{$}\NormalTok{Year }\SpecialCharTok{\textless{}=} \DecValTok{2016}\NormalTok{,]}



\NormalTok{Last5YearData[Last5YearData}\SpecialCharTok{==}\StringTok{"N/A"}\NormalTok{] }\OtherTok{=} \ConstantTok{NA} 

\NormalTok{Last5YearData }\OtherTok{=}\NormalTok{ Last5YearData[}\FunctionTok{complete.cases}\NormalTok{(Last5YearData}\SpecialCharTok{$}\NormalTok{Year),]}
\FunctionTok{boxplot}\NormalTok{(Global\_Sales}\SpecialCharTok{\textasciitilde{}}\NormalTok{Platform,Last5YearData)}
\end{Highlighting}
\end{Shaded}

\includegraphics{Test_files/figure-latex/unnamed-chunk-14-1.pdf}

\begin{Shaded}
\begin{Highlighting}[]
\FunctionTok{boxplot}\NormalTok{(Global\_Sales}\SpecialCharTok{\textasciitilde{}}\NormalTok{Genre,Last5YearData)}
\end{Highlighting}
\end{Shaded}

\includegraphics{Test_files/figure-latex/unnamed-chunk-14-2.pdf}

Zbog outliera s vrijednostima koje previše odstupaju od box-plota
izbacit cemo 5\% najvecih vrijednosti kako bi graf postao čitljiviji.

\begin{Shaded}
\begin{Highlighting}[]
\NormalTok{TrimmedLast5YearData }\OtherTok{=}\NormalTok{ Last5YearData[Last5YearData}\SpecialCharTok{$}\NormalTok{Global\_Sales }\SpecialCharTok{\textless{}} \FunctionTok{quantile}\NormalTok{(Last5YearData}\SpecialCharTok{$}\NormalTok{Global\_Sales, }\FloatTok{0.95}\NormalTok{),]}

\FunctionTok{boxplot}\NormalTok{(}\AttributeTok{las =} \DecValTok{2}\NormalTok{,Global\_Sales}\SpecialCharTok{\textasciitilde{}}\NormalTok{Platform,TrimmedLast5YearData)}
\end{Highlighting}
\end{Shaded}

\includegraphics{Test_files/figure-latex/unnamed-chunk-15-1.pdf}

\begin{Shaded}
\begin{Highlighting}[]
\FunctionTok{boxplot}\NormalTok{(}\AttributeTok{las =} \DecValTok{2}\NormalTok{,}\AttributeTok{cex.names=}\NormalTok{.}\DecValTok{5}\NormalTok{,Global\_Sales}\SpecialCharTok{\textasciitilde{}}\NormalTok{Genre,TrimmedLast5YearData)}
\end{Highlighting}
\end{Shaded}

\includegraphics{Test_files/figure-latex/unnamed-chunk-15-2.pdf}

Iz ovih Box plot-ova možemo uvidjeti da se u modernije vrijeme najveća
prodaja primjeraka događa na konsolama PS3,PS4,X360 i XOne dok je
prodaja jako slaba na PSP-u i PSV-i. Od žanrova najprodavanije su
Platform,Shooter i Sports dok Adventure i Puzzle poprilično sigurno drže
zadnja mjesta.

Zbog ovoliko velikih razlika između platformi/žanrova smatramo da odabir
istih poprilično utječe na prodaju te se do neke mjere može predvidjeti.

Linearnom regresijom se predviđa vrijednost varijable izlaza Y
(Global\_Sales) obzirom na varijable unosa X (Genre,Platform).

\begin{Shaded}
\begin{Highlighting}[]
\NormalTok{lrmodel }\OtherTok{\textless{}{-}} \FunctionTok{lm}\NormalTok{(Global\_Sales }\SpecialCharTok{\textasciitilde{}}\NormalTok{ Genre }\SpecialCharTok{+}\NormalTok{ Platform,TrimmedLast5YearData)}
\FunctionTok{summary}\NormalTok{(lrmodel)}
\end{Highlighting}
\end{Shaded}

\begin{verbatim}
## 
## Call:
## lm(formula = Global_Sales ~ Genre + Platform, data = TrimmedLast5YearData)
## 
## Residuals:
##      Min       1Q   Median       3Q      Max 
## -0.62678 -0.22882 -0.09646  0.06797  1.69579 
## 
## Coefficients:
##                   Estimate Std. Error t value Pr(>|t|)    
## (Intercept)        0.25044    0.02224  11.260  < 2e-16 ***
## GenreAdventure    -0.12318    0.02632  -4.681 3.01e-06 ***
## GenreFighting     -0.03165    0.03928  -0.806 0.420552    
## GenreMisc         -0.03996    0.03192  -1.252 0.210715    
## GenrePlatform      0.14519    0.04504   3.224 0.001281 ** 
## GenrePuzzle       -0.07094    0.07348  -0.965 0.334437    
## GenreRacing        0.00804    0.03914   0.205 0.837277    
## GenreRole-Playing  0.05377    0.02412   2.230 0.025846 *  
## GenreShooter       0.19331    0.03132   6.171 7.83e-10 ***
## GenreSimulation   -0.08501    0.04822  -1.763 0.078025 .  
## GenreSports        0.06954    0.02770   2.511 0.012100 *  
## GenreStrategy     -0.05711    0.04787  -1.193 0.233023    
## PlatformDS        -0.08927    0.06820  -1.309 0.190701    
## PlatformPC        -0.05388    0.03298  -1.634 0.102448    
## PlatformPS3        0.09128    0.02705   3.375 0.000750 ***
## PlatformPS4        0.10526    0.02996   3.513 0.000451 ***
## PlatformPSP       -0.15645    0.03549  -4.408 1.09e-05 ***
## PlatformPSV       -0.10016    0.02792  -3.587 0.000341 ***
## PlatformWii        0.10574    0.05834   1.813 0.070020 .  
## PlatformWiiU       0.07170    0.03838   1.868 0.061882 .  
## PlatformX360       0.19303    0.03124   6.179 7.46e-10 ***
## PlatformXOne       0.11353    0.03442   3.299 0.000985 ***
## ---
## Signif. codes:  0 '***' 0.001 '**' 0.01 '*' 0.05 '.' 0.1 ' ' 1
## 
## Residual standard error: 0.378 on 2582 degrees of freedom
## Multiple R-squared:  0.1231, Adjusted R-squared:  0.116 
## F-statistic: 17.26 on 21 and 2582 DF,  p-value: < 2.2e-16
\end{verbatim}

\(R^2\) od modela nam govori da je model zaslužan samo za 11.5\%
varijacije u podacima što nije pretežito velik broj, ali sudeći da je
bitnija kvaliteta igrice nego samo kategorizacija kod prodaje smatramo
ga prihvatljivim.

Dodavanjem Izdavača i godinu izdavanja u regresiju pretpostavljamo da će
se rezultat poboljšati. Ovdje ćemo zbog prevelikog broj koeficijenata u
ispisu samo ispisat platforme i žanr.

\begin{Shaded}
\begin{Highlighting}[]
\NormalTok{lrmodel }\OtherTok{=} \FunctionTok{lm}\NormalTok{(Global\_Sales }\SpecialCharTok{\textasciitilde{}}\NormalTok{ Platform }\SpecialCharTok{+}\NormalTok{ Genre }\SpecialCharTok{+}\NormalTok{ Year }\SpecialCharTok{+}\NormalTok{ Publisher,TrimmedLast5YearData)}
\FunctionTok{my.summary.lm}\NormalTok{(}\FunctionTok{summary}\NormalTok{(lrmodel), }\AttributeTok{my.rows=}\DecValTok{1}\SpecialCharTok{:}\DecValTok{22}\NormalTok{)}
\end{Highlighting}
\end{Shaded}

\begin{verbatim}
## 
## Call:
## lm(formula = Global_Sales ~ Platform + Genre + Year + Publisher, 
##     data = TrimmedLast5YearData)
## 
## Residuals:
##      Min       1Q   Median       3Q      Max 
## -0.79200 -0.17261 -0.02958  0.08276  1.55052 
## 
## Coefficients:
##                   Estimate Std. Error t value Pr(>|t|)    
## (Intercept)        0.15403    0.06604   2.332 0.019758 *  
## PlatformDS        -0.10219    0.06779  -1.507 0.131841    
## PlatformPC        -0.08570    0.03422  -2.504 0.012342 *  
## PlatformPS3        0.12752    0.02690   4.741 2.26e-06 ***
## PlatformPS4        0.22572    0.03121   7.232 6.38e-13 ***
## PlatformPSP       -0.09025    0.03784  -2.385 0.017143 *  
## PlatformPSV        0.01532    0.02887   0.531 0.595670    
## PlatformWii        0.01962    0.05407   0.363 0.716745    
## PlatformWiiU      -0.06006    0.03527  -1.703 0.088711 .  
## PlatformX360       0.15224    0.03088   4.930 8.78e-07 ***
## PlatformXOne       0.11790    0.03515   3.354 0.000808 ***
## GenreAdventure    -0.02315    0.02883  -0.803 0.422065    
## GenreFighting      0.05013    0.03669   1.366 0.171941    
## GenreMisc         -0.04667    0.03049  -1.531 0.126017    
## GenrePlatform      0.08888    0.04418   2.012 0.044366 *  
## GenrePuzzle       -0.15241    0.07472  -2.040 0.041485 *  
## GenreRacing        0.07438    0.04801   1.549 0.121467    
## GenreRole-Playing  0.08845    0.02412   3.668 0.000250 ***
## GenreShooter       0.13326    0.03100   4.298 1.79e-05 ***
## GenreSimulation   -0.04879    0.04943  -0.987 0.323740    
## GenreSports       -0.01423    0.03032  -0.469 0.638810    
## GenreStrategy     -0.03178    0.04469  -0.711 0.477027    
## ---
## Signif. codes:  0 '***' 0.001 '**' 0.01 '*' 0.05 '.' 0.1 ' ' 1
## 
## Residual standard error: 0.3307 on 2395 degrees of freedom
##   (6 observations deleted due to missingness)
## Multiple R-squared:  0.3771, Adjusted R-squared:  0.3246 
## F-statistic: 7.177 on 202 and 2395 DF,  p-value: < 2.2e-16
\end{verbatim}

S novim \(R^2\) od 0.35 smatramo da imamo rezultat s kojim smo
zadovoljni. Još trebamo provjeriti važe li pretpostavke regresije.

\begin{Shaded}
\begin{Highlighting}[]
\FunctionTok{hist}\NormalTok{((lrmodel}\SpecialCharTok{$}\NormalTok{residuals))}
\end{Highlighting}
\end{Shaded}

\includegraphics{Test_files/figure-latex/unnamed-chunk-19-1.pdf}

\begin{Shaded}
\begin{Highlighting}[]
\FunctionTok{qqnorm}\NormalTok{(}\FunctionTok{rstandard}\NormalTok{(lrmodel))}
\FunctionTok{qqline}\NormalTok{(}\FunctionTok{rstandard}\NormalTok{(lrmodel))}
\end{Highlighting}
\end{Shaded}

\includegraphics{Test_files/figure-latex/unnamed-chunk-19-2.pdf}

Na ova dva grafa vidimo kao što smo i prije vidjeli da su podaci o
prodajama eskponencijalno distribuirani pa odudaraju na jednom kraju
kvantil-kvantil plota. To nam daje ideju da možemo dobiti bolji model
predikcijom nakon što normaliziramo podatke logaritmom.

\begin{Shaded}
\begin{Highlighting}[]
\NormalTok{expmodel }\OtherTok{=} \FunctionTok{lm}\NormalTok{(}\FunctionTok{log}\NormalTok{(Global\_Sales) }\SpecialCharTok{\textasciitilde{}}\NormalTok{ Platform }\SpecialCharTok{+}\NormalTok{ Genre }\SpecialCharTok{+}\NormalTok{ Year }\SpecialCharTok{+}\NormalTok{ Publisher,TrimmedLast5YearData)}

\FunctionTok{hist}\NormalTok{((expmodel}\SpecialCharTok{$}\NormalTok{residuals))}
\end{Highlighting}
\end{Shaded}

\includegraphics{Test_files/figure-latex/unnamed-chunk-20-1.pdf}

\begin{Shaded}
\begin{Highlighting}[]
\FunctionTok{qqnorm}\NormalTok{(}\FunctionTok{rstandard}\NormalTok{(expmodel))}
\FunctionTok{qqline}\NormalTok{(}\FunctionTok{rstandard}\NormalTok{(expmodel))}
\end{Highlighting}
\end{Shaded}

\includegraphics{Test_files/figure-latex/unnamed-chunk-20-2.pdf}

\begin{Shaded}
\begin{Highlighting}[]
\FunctionTok{my.summary.lm}\NormalTok{(}\FunctionTok{summary}\NormalTok{(expmodel), }\AttributeTok{my.rows =} \DecValTok{1}\SpecialCharTok{:}\DecValTok{22}\NormalTok{)}
\end{Highlighting}
\end{Shaded}

\begin{verbatim}
## 
## Call:
## lm(formula = log(Global_Sales) ~ Platform + Genre + Year + Publisher, 
##     data = TrimmedLast5YearData)
## 
## Residuals:
##     Min      1Q  Median      3Q     Max 
## -4.0397 -0.6017  0.0225  0.6371  3.4409 
## 
## Coefficients:
##                    Estimate Std. Error t value Pr(>|t|)    
## (Intercept)       -2.102067   0.201490 -10.433  < 2e-16 ***
## PlatformDS        -0.476447   0.206844  -2.303 0.021341 *  
## PlatformPC        -0.531619   0.104418  -5.091 3.83e-07 ***
## PlatformPS3        0.283182   0.082076   3.450 0.000570 ***
## PlatformPS4        0.672991   0.095231   7.067 2.07e-12 ***
## PlatformPSP       -0.527777   0.115444  -4.572 5.08e-06 ***
## PlatformPSV       -0.007219   0.088079  -0.082 0.934688    
## PlatformWii       -0.076154   0.164979  -0.462 0.644413    
## PlatformWiiU      -0.171927   0.107608  -1.598 0.110237    
## PlatformX360       0.314255   0.094222   3.335 0.000865 ***
## PlatformXOne       0.295497   0.107246   2.755 0.005908 ** 
## GenreAdventure    -0.164697   0.087972  -1.872 0.061307 .  
## GenreFighting      0.353267   0.111951   3.156 0.001622 ** 
## GenreMisc         -0.070063   0.093030  -0.753 0.451450    
## GenrePlatform      0.117522   0.134805   0.872 0.383407    
## GenrePuzzle       -0.707193   0.227978  -3.102 0.001944 ** 
## GenreRacing       -0.011275   0.146494  -0.077 0.938656    
## GenreRole-Playing  0.404544   0.073583   5.498 4.25e-08 ***
## GenreShooter       0.357657   0.094595   3.781 0.000160 ***
## GenreSimulation    0.028772   0.150822   0.191 0.848721    
## GenreSports       -0.040151   0.092505  -0.434 0.664293    
## GenreStrategy     -0.147988   0.136348  -1.085 0.277870    
## ---
## Signif. codes:  0 '***' 0.001 '**' 0.01 '*' 0.05 '.' 0.1 ' ' 1
## 
## Residual standard error: 1.009 on 2395 degrees of freedom
##   (6 observations deleted due to missingness)
## Multiple R-squared:  0.5356, Adjusted R-squared:  0.4965 
## F-statistic: 13.68 on 202 and 2395 DF,  p-value: < 2.2e-16
\end{verbatim}

Lako vidimo da smo dobili puno bolju normalnost reziduala kao i puno
bolji fit podacima sa R\^{}2 od 0.4965.

\begin{Shaded}
\begin{Highlighting}[]
\CommentTok{\# izvuci koeficijente u data frame}
\NormalTok{coeffs }\OtherTok{=} \FunctionTok{data.frame}\NormalTok{(}
  \AttributeTok{keyName=}\FunctionTok{names}\NormalTok{(lrmodel}\SpecialCharTok{$}\NormalTok{coefficients), }
  \AttributeTok{value =}\NormalTok{ lrmodel}\SpecialCharTok{$}\NormalTok{coefficients, }
  \AttributeTok{row.names=}\ConstantTok{NULL}
\NormalTok{)}

\NormalTok{coeffs\_asc }\OtherTok{=}\NormalTok{ coeffs[}\FunctionTok{order}\NormalTok{(coeffs}\SpecialCharTok{$}\NormalTok{value),]}
\FunctionTok{slice\_min}\NormalTok{(coeffs\_asc, }\AttributeTok{order\_by =}\NormalTok{ coeffs\_asc}\SpecialCharTok{$}\NormalTok{value, }\AttributeTok{n=}\DecValTok{10}\NormalTok{)}
\end{Highlighting}
\end{Shaded}

\begin{verbatim}
##                                      keyName      value
## 1                              PublisherMoss -0.4459947
## 2                              PublisherCave -0.3661340
## 3               PublisherTopWare Interactive -0.3311994
## 4                PublisherFalcom Corporation -0.3151810
## 5                          PublisherSystem 3 -0.3088876
## 6  PublisherCloud Imperium Games Corporation -0.3040476
## 7                          PublisherEA Games -0.2648139
## 8        PublisherInterworks Unlimited, Inc. -0.2573205
## 9                              PublisherYeti -0.2399038
## 10                     PublisherMastertronic -0.2380293
\end{verbatim}

\begin{Shaded}
\begin{Highlighting}[]
\FunctionTok{slice\_max}\NormalTok{(coeffs\_asc, }\AttributeTok{order\_by =}\NormalTok{ coeffs\_asc}\SpecialCharTok{$}\NormalTok{value, }\AttributeTok{n=}\DecValTok{10}\NormalTok{)}
\end{Highlighting}
\end{Shaded}

\begin{verbatim}
##                                            keyName     value
## 1                             PublisherHello Games 1.4466005
## 2                               Publishermixi, Inc 0.8329694
## 3                                PublisherNintendo 0.5789624
## 4                                     PublisherTHQ 0.5528769
## 5                      PublisherBethesda Softworks 0.5168896
## 6                  PublisherMicrosoft Game Studios 0.5134824
## 7                     PublisherActivision Blizzard 0.4701940
## 8                                  PublisherMojang 0.4665832
## 9  PublisherWarner Bros. Interactive Entertainment 0.4346224
## 10                        PublisherElectronic Arts 0.4168079
\end{verbatim}

Vidimo da su top 10 najvećih i najmanjih koeficijenata svi u ovisnosti
od publishera. To ima smisla jer žanr, platforma i godina izdavanja igre
ne govori puno o kvaliteti igre. Publisheri već imaju određenu
popularnost i reputaciju te se u puno slučajeva može predvidjeti
kvaliteta igre samo kroz publishera dok se to uglavnom ne može ako su u
pitanju samo žanr i platforma.

\hypertarget{pitanje-zamislite-da-radite-videoigru.-kakve-karakteristike-bi-ta-igra-trebala-imati-ako-ux17eelite-da-ona-bude-ux161to-prodavanija-u-odreux111enoj-regiji}{%
\subsection{4. Pitanje: Zamislite da radite videoigru. Kakve
karakteristike bi ta igra trebala imati ako želite da ona bude što
prodavanija u određenoj
regiji?}\label{pitanje-zamislite-da-radite-videoigru.-kakve-karakteristike-bi-ta-igra-trebala-imati-ako-ux17eelite-da-ona-bude-ux161to-prodavanija-u-odreux111enoj-regiji}}

\end{document}
